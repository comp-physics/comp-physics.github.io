%!TEX-root = cv.tex
%!TEX TS-program = pdflatex

\documentclass{article}

\usepackage[left=1.5in,right=1.5in,top=1in,bottom=1in]{geometry}
\usepackage{comment}
\usepackage{tabu}
\usepackage{booktabs}
\usepackage{xstring}
\usepackage{etoolbox}
\usepackage[backend=bibtex,defernumbers=true,sorting=ydnt,style=authoryear,doi=true,maxnames=99,url=true]{biblatex}
\usepackage{calc}
\usepackage{comment}
\usepackage{microtype}
\usepackage{hyphenat}
\usepackage{changepage}
\usepackage{multicol}
\usepackage{xspace}
\usepackage{enumitem}
\usepackage{hyperref}
\usepackage[]{xcolor}

\definecolor{lightblue}{rgb}{0.63, 0.74, 0.78}
\definecolor{seagreen}{rgb}{0.18, 0.42, 0.41}
\definecolor{orange}{rgb}{0.85, 0.55, 0.13}
\definecolor{silver}{rgb}{0.69, 0.67, 0.66}
\definecolor{rust}{rgb}{0.72, 0.26, 0.06}

\colorlet{lightsilver}{silver!30!white}
\colorlet{darkorange}{orange!75!black}
\colorlet{darksilver}{silver!65!black}
\colorlet{darklightblue}{lightblue!65!black}
\colorlet{darkrust}{rust!85!black}


\hypersetup{
  colorlinks=true,
  linkcolor=silver,
  citecolor=silver,
  urlcolor=seagreen,
  pdfauthor=author,
}

\newcommand\GIT{\mbox{Georgia Institute of Technology}\xspace}  
\newcommand\GT{\mbox{Georgia Tech}\xspace}  
\newcommand\CIT{\mbox{California Institute of Technology}\xspace}  
\newcommand\MIT{\mbox{Massachusetts Institute of Technology}\xspace}  
\newcommand\UIUC{\mbox{University of Illinois at Urbana--Champaign}\xspace}  
\newcommand\UMD{\mbox{University of Michigan--Dearborn}\xspace}

\newcommand\Florian{\mbox{F. Sch\"{a}fer}\xspace}
\newcommand\Rich{\mbox{R. Vuduc}\xspace}

\newcommand\APBC{\mbox{Assistant Professor by Courtesy (0\%)}\xspace}

% Remove indent spacing
\setlength{\parindent}{0in}

% Filters for bib
\addbibresource{ref.bib}

\defbibfilter{other}{
  type=report or
  type=thesis
}

\defbibfilter{invited}{
  type=incollection and
  keyword=invited
}

\defbibfilter{talk}{
  type=incollection and
  not keyword=invited
}

\defbibfilter{heavy}{
  type=inproceedings and
  keyword=heavyref
}

\defbibfilter{nonheavy}{
  type=inproceedings and
  not keyword=heavyref
}

\defbibfilter{fullpapers}{
    type=article or keyword=heavyref
}

% Bold names
\newboolean{bold}
\newcommand{\makeauthorsbold}[1]{%
  \DeclareNameFormat{author}{%
  \setboolean{bold}{false}%
    \renewcommand{\do}[1]{\expandafter\ifstrequal\expandafter{\namepartfamily}{####1}{\setboolean{bold}{true}}{}}%
    \docsvlist{#1}%
    \ifthenelse{\value{listcount}=1}
    {%
     {\expandafter\ifthenelse{\boolean{bold}}{\mkbibbold{\namepartfamily\addcomma\addspace \namepartgiveni}}{\namepartfamily\addcomma\addspace \namepartgiveni}}%
    }{\ifnumless{\value{listcount}}{\value{liststop}}
      {\expandafter\ifthenelse{\boolean{bold}}{\mkbibbold{\addcomma\addspace \namepartfamily\addcomma\addspace \namepartgiveni}}{\addcomma\addspace \namepartfamily\addcomma\addspace \namepartgiveni}}%
      {\expandafter\ifthenelse{\boolean{bold}}{\mkbibbold{\addcomma\addspace \namepartfamily\addcomma\addspace \namepartgiveni\addcomma\isdot}}{\addcomma\addspace \namepartfamily\addcomma\addspace \namepartgiveni\addcomma\isdot}}%
      }
    \ifthenelse{\value{listcount}<\value{liststop}}
    {\addcomma\space}{}
  }
}
\makeauthorsbold{Le Berre,Bati,Chrit,Lee,Kocherla,Radhakrishnan,Arias,Song,Wilfong,Hawkins,Yu,Panchal,Zeng,Kothari,Sinha,Chu,Manasreh,Gupta,Adam,{*}Wilfong,{*}Radhakrishnan,{*Le Berre}}

% Get reverse numbering
\makeatletter
\patchcmd{\blx@printbibliography}
  {\blx@bibliography\blx@tempa}
  {\setcounter{bibitemtotal}{0}%
   \begingroup
   \def\do##1{\stepcounter{bibitemtotal}}%
   \dolistloop{\blx@tempa}%
   \endgroup
   \blx@bibliography\blx@tempa}{}{}
\makeatother

\newcounter{bibitemtotal}
\newrobustcmd*{\mkbibdesc}[1]{%
  \number\numexpr\value{bibitemtotal}+1-#1\relax}
\DeclareFieldFormat{labelnumber}{\mkbibdesc{#1}}
\DeclareFieldFormat{labelnumberwidth}{\mkbibbrackets{#1}}

\defbibenvironment{bibliography}
  {\list
     {\printtext[labelnumberwidth]{\printfield{labelprefix}\printfield{labelnumber}}}
     {}%
      \renewcommand*{\makelabel}[1]{\hss##1}}
  {\endlist}
  {\item}

% Set spacing in itemize
\setlist{itemsep=3pt,topsep=3pt,parsep=0pt,partopsep=0pt} 

% Set spacing around titles (syntax: spacing left, before, after)
\usepackage{titlesec}
\titlespacing*{\section}
{0pt}{6pt}{6pt}
\titlespacing*{\subsection}
{0pt}{6pt}{6pt}

\usepackage{lastpage}
\usepackage{fancyhdr}
\pagestyle{fancy}
\renewcommand{\headrulewidth}{0pt}

\lhead{}\chead{\color{silver}CV Compiled on: \today}\rhead{}
\cfoot{}
\rfoot{\color{silver}Page \thepage\xspace of \pageref{LastPage}}
\cfoot{\color{silver}Latest at \href{https://comp-physics.group/cv/cv.pdf}{\tt comp-physics.group/cv/cv.pdf}}
\lfoot{\color{silver}Spencer H. Bryngelson}

% \usepackage{titletoc}
% \setcounter{secnumdepth}{1}
% \setcounter{tocdepth}{1}

\usepackage{csvsimple}
\usepackage{longtable}


\usepackage[T1]{fontenc}
\usepackage{ebgaramond}
% \usepackage{garamondlibre}
% \usepackage{garamondx}
% \usepackage{CormorantGaramond}

%% For TOC
\usepackage{tocloft}               % For table of contents formatting
\usepackage{setspace}              % For spacing
% \setcounter{tocdepth}{1}

%% Table of contents formatting
% \renewcommand\cftsecafterpnum{\vskip 0pt} % for spacing after each entry
% \renewcommand\cftsubsecafterpnum{\vskip 0pt} % for spacing after each entry
% \renewcommand\cftsubsubsecafterpnum{\vskip 5pt} % for spacing after each entry
% \setlength{\cftbeforetoctitleskip}{0in}


% \usepackage{titletoc}
% \setcounter{secnumdepth}{1}
% \setcounter{tocdepth}{1}

\usepackage{csvsimple}
\usepackage{longtable}


\usepackage[T1]{fontenc}
\usepackage{ebgaramond}
% \usepackage{garamondlibre}
% \usepackage{garamondx}
% \usepackage{CormorantGaramond}

% \begin{filecontents*}{grade.csv}
% name,givenname,matriculation,gender,grade
% Maier,Hans,12345,m,1.0
% Huber,Anna,23456,f,2.3
% Weisbaeck,Werner,34567,m,5.0
% \end{filecontents*}


% \begin{filecontents*}{grade2.csv}
% description,name
% Resources for learning about numerical methods.,awesome-numerics
% Website of the Computational Physics Group @ GT,comp-physics.github.io
% 3D Spectral boundary integral solver for cell-scale blood flow,RBC3D
% PyQBMMlib is a Python extension of QBMMlib. ,PyQBMMlib
% Integrate bubble dynamics faster!,bubble-dynamics-with-multiscale-resnets
% modal decomposition via high-order statistics for people,tensor-modal-decomp
% 2D Spectral boundary integral solver for cell-scale blood flow,RBC2D
% Benchmarking FVMs on different hardware and under different optimizations,fvm-risc
% Mathematica package for quadrature-based moment methods and population balance equations.,QBMMlib
% Inertial Microcavitation Rheometry,IMR
% A WENO solver for 1D scalar PDEs,WENO-scalar
% A modified WENO method that improves interface sharpness via neural networks.,WENO-NN
% A version of ECOGEN that was developed and used at Caltech,ECOGEN-CIT
% 1D Ensemble-averaging solver for dilute cavitating bubbly flows. Finite volume with WENO/Riemann solvers.,EnsAvg_1D_Tait
% Dynamics of cavitating bubble populations,PyCav
% A shock-capturing adjoint solver for the compressible flow equations,1D-Shocks-Adjoint-Euler-Solver
% A solver for the eigenmodes of an unstable viscoelastic jet,capillary_instability
% \end{filecontents*}

\begin{document}

\begin{center}
    {\LARGE \bf Spencer H. Bryngelson} 
\end{center}

% \tableofcontents

\section{Basic information}
\begin{itemize}
    \item[] \textbf{Title:} Assistant Professor, School of Computational Science \& Engineering
    \item[] \textbf{Institution:} \GIT
    \item[] \textbf{Address:} S1313 CODA, 756 W Peachtree St NW, Atlanta, GA 30308
    \item[] \textbf{Email:} \href{mailto:shb@gatech.edu}{\texttt{shb@gatech.edu}}
    \item[] \textbf{Website:} \href{https://comp-physics.group}{\texttt{https://comp-physics.group}}
\end{itemize}

\section{Education}

\begin{itemize}
    \item \UIUC
    \begin{itemize}
        \item[] (2017) Doctor of Philosophy, Theoretical \& Applied Mechanics
        \item[] (2015) Master of Science, Theoretical \& Applied Mechanics
        \item[] (2015) Graduate Certificate, Computational Science \& Engineering
    \end{itemize}
    \item \UMD
    \begin{itemize}
        \item[] (2013) Bachelor of Science, Mechanical Engineering
        \item[] (2013) Bachelor of Science, Engineering Mathematics
    \end{itemize}
\end{itemize}


\section{Positions held}

\begin{itemize}
    \item (2021--Present) Assistant Professor, School of Computational Science \& Engineering, College of Computing, \GIT
    \item (2022) Visiting Scholar, Stanford University, Center for Turbulence Research (Summer Program)
    \item (2018--21) Senior Postdoctoral Scholar, \CIT, with Tim Colonius
    \item (2019) Visiting Researcher, \MIT, with Themis Sapsis
    \item (2017--18) Postdoctoral Researcher, XPACC (PSAAP II center), with Carlos Pantano, Dan Bodony, Jon Freund
    \item (2013--17) Graduate Research Fellow, \UIUC, with Jon Freund
    \item (2015) Alumni Teaching Fellow, \UIUC
    \item (2012--13) Undergraduate Research Assistant, \UMD, with Eric Ratts
\end{itemize}

\section{Teaching}

\subsection{\GIT}

\begin{center}
    \begin{tabular}{ r l l c c }
        \hline\hline
        \bf Semester  &\bf Number & \bf Course Title & \bf Students & \bf TAT \\
        \hline
        Spring 2023 & CSE6730 & Modeling \& Simulation & n/a & n/a \\
        Fall 2022 & VIP[2/3/4]60[1/2]  & Team Phoenix Cluster Competition Team & 17 & n/a \\
        Fall 2022 & CX/MATH4640 & Numerical Analysis I & 36 & n/a \\
        Fall 2021 & CX/MATH4640 & Numerical Analysis I & 43 & Yes \\
        \hline\hline
    \end{tabular}
\end{center}
$\ast$\textbf{TAT:} Thank a Teacher Award received

\subsection{Other Institutions}

\begin{center}
    \begin{tabular}{ r l l c l }
        \hline\hline
        \bf Semester  &\bf Number & \bf Course Title & \bf Students & \bf Institute \\
        \hline
        Fall   2015 & ME 310  & Fundamentals of Fluid Dynamics & 82 & UIllinois \\
        Fall   2013 & ME 3601 & Design and Analysis of Machine Elements & 35 & UMichigan\\
        Spring 2012 & ME 364  & Probability, Statistics, and Reliability in Design & 32 & UMichigan \\
        Fall   2012 & ME 230  & Statics and Mechanics of Materials & 61 & UMichigan \\
        \hline\hline
    \end{tabular}
\end{center}


\section{Students}

\subsection{Graduate}

\begin{itemize}
    \item Jesus Arias, Ph.D.\ student (CSE, co-advised with L.\ Sankar)
    \item Fatima Ezahra Chrit, Ph.D.\ student (ME and CSE, co-advised with A.\ Alexeev)
    \item Anand Radhakrishnan, Ph.D.\ student (CSE)
    \item Nathanael Gutierrez (CS)
    \item Anshuman Sinha, M.S.\ student (CSE)
    \item Zhixin Song, Ph.D.\ Student (Physics)
    \item Benjamin Wilfong, Ph.D.\ student (CSE)
    \item Haocheng Yu, Ph.D.\ student (CSE, co-advised with K.\ Ahuja)
\end{itemize}

\subsection{Undergraduate}

\begin{itemize}
    \item Ajay Bati (CS)
    \item Ansh Gupta (CS)
    \item Arjun Bhamra (CS)
    \item Sriharsha Kocherla (CS)
    \item Yash Kothari (CS)
    \item Henry Le Berre (CS)
    \item Qi Zeng (CS and Math, co-advised with \Florian)
\end{itemize}

\section{Awards}

\begin{itemize}
    \item (2022) Ralph E. Powe Junior Faculty Enhancement Award, Oak Ridge National Lab
    \item (2022--23) \GT Faculty Writing Scholar
    \item (2022--23) Class of 1969 Teaching Fellow, \GIT
    \item (2017) Stanley Weiss Outstanding Dissertation Award, \UIUC
    \item (2016) Hassan Aref Award (research in fluid mechanics), \UIUC
    \item (2015) Alumni Teaching Fellowship, \UIUC
    \item (2010--13) Dean's List, \UMD
    \item (2011) Pi Tau Sigma (honor society, member), \UMD
\end{itemize}

\section{Research support}

\subsection{Funded grants}

\begin{itemize}
    \item (2023) PI: DOE/Sandia National Laboratory (subcontract), ``\textit{Vibrated bubbly flow simulation}'' ($\$65$K)
    \item (2022--23) PI: DOE ORAU Powe, ``\textit{A methodologically coherent multi-scale model for multiphase flow}'' ($\$10$K)
    \item (2022--26) PI: DOD ONR N000142212519, ``\textit{Stochastic framework for cavitating flows: mesoscale modeling and acceleration}'' ($\$560$K)
    \item (2022--23) co-PI: GTRI IRAD, ``\textit{Quantum optimization for lattice Boltzmann simulation (QOLBS)}'' ($\$40$K), PI B. Gard (GTRI)
    \item (2022) PI: GT Seed Grant, Forming Teams ``\textit{Quantum computing for next-generation engineering simulation}'' ($\$50$K)
    \item (2022) PI: GTQA DE00013211, ``\textit{Quantum algorithms for lattice Boltzmann fluid flow simulation}'' ($\$14.5$K)
\end{itemize}

\subsection{Funded resource and hardware awards}

\begin{itemize}
    \item (2021--23) PI: Oak Ridge National Lab CFD154, Director's Discretionary, ``\textit{Accelerated sub-grid multi-component flow physics}'' (20K node hours)
    \item (2022) PI: NVIDIA Academic Hardware Grant Program (4x BlueField-2 E-Series DPU, $\$12$K value)
    \item (2022) PI: Georgia Tech Tech.\ Fee ``\textit{ARM HPC Dev Kits for next-generation supercomputing}'' (10 NVIDIA ARM HPC Dev.\ Kits, $\$240$K value)
    \item (2022) PI: AMD MI200-series GPU Server ($\$77$K value)
    \item (2022) PI: Stanford CTR Summer Program ``\textit{Fast macroscopic forcing for operator recovery via locality and causality with application to compressible and multiphase flow}'' ($\$8$K, with \Florian, SHB share $\$4$K)
    \item (2022) PI: NVIDIA Academic Hardware Grant Program (2x A100 80GB PCIe GPUs, $\$30$K value)
    \item (2021--22) PI: XSEDE TG-PHY210084, ``\textit{High-fidelity simulation of high-speed flowing dispersions via a stochastic sub-grid model}''  (200K Node Hours, $\$30$K value)
    \item (2019--20) co-PI: XSEDE TG-CTS120005, ``\textit{Advanced immersed boundary and interface-capturing methods for simulations of complex flows}'' (9M Node Hours, $\$1.35$M value)
\end{itemize}

% \subsection{Pending grants}

\begin{itemize}
    \item (2023--24) PI: ASME Hawthornwaite Research Inititation Grant ($\$20$K)
    \item (2023--25) PI: DOE INCITE, ``\textit{The first full-resolution liquid--gas disperse flow simulations}'' (500K Summit Node Hours, 2M Frontier Node Hours, $\$$1M value)
\end{itemize}


\subsection{Rejected proposalas}

\begin{itemize}
    \item (2022--25) co-PI: NSF OAC CORE 21-616, ``\textit{Enabling rapid, targeted optimization of PDE solvers via hardware--software coupled autotuning for novel architectures}'' (SHB share: $\$300$K, Total: $\$600$K)
    \item (2022--23) co-PI: NOAA SBIR OAR-TPO 2007117 ``\textit{Using bubbles to reduce underwater noise from ships and ferries}'' (SHB share: $\$39$K, Total: $\$125$K)
    \item (2022) PI: Georgia Tech TechFee ``\textit{ARM HPC Dev Kits for next-generation supercomputing}'' (8x NVIDIA HPC Dev. Kits, $\$700$K value)
    \item (2022--24) PI: DOE ASCR DE-FOA-0002717, ``\textit{Split, Prune, Unify: A quantum algorithm development strategy}'' (SHB share: $\$200$K, Total: $\$400$K)
    \item (2022--24) co-PI: DOE ASCR DE-FOA-0002717, ``\textit{Sharing and combining SciML models, robustly and securely}'' (SHB share: $\$200$K, Total: $\$400$K)
    \item (2023--24) PI: Google Collabs, ``\textit{Enabling subgrid-scale modeling of multiphase flows using differentiable CFD and machine learning models}'' ($\$90$K)
\end{itemize}


\section{Professional activity}

\subsection{Appointments and memberships}

\begin{itemize}
    \item (2021--Present) NATO Science \& Technology Organization, Technical Team Member
    \item (2015--Present) Society of Industrial and Applied Mathematics, Member
    \item (2014--Present) American Physical Society, Member
\end{itemize}

\subsection{Referee}

\begin{itemize}
    \item AIAA Journal
    \item Fluids
    \item IEEE International Parallel \& Distributed Processing Symposium
    \item International Journal of Multiphase Flow
    \item International Journal of Offshore and Polar Engineering
    \item Journal of Computational Physics
    \item Journal of Fluid Mechanics
    \item Physical Review E
    \item Physical Review Fluids
    \item PLOS Computational Biology
    \item Symposium of Naval Hydrodynamics
    \item Theoretical and Computational Fluid Dynamics
\end{itemize}

\section{Service and outreach}

\subsection{Georgia Tech}

\subsubsection{Institute-level}

\begin{itemize}
    \item (2021--Present) Georgia Tech \textit{HPC Hackathon}, initiator and organizer, recruited sponsors Oak Ridge National Lab and NVIDIA 
    \item (2022--Present) Georgia Tech \textit{Scientific Software Engineering Center}, Advisory Board
    \item (2022--Present) PURA Award Reviewer
    \item (2022--Present) ORAU Powe Award Reviewer
    \item (2022) Faculty Search Panel, Professional Development Workshops, Georgia Tech Center for Teaching and Learning
\end{itemize}

\subsubsection{CoC-level}

\begin{itemize}
    \item (2022--Present) VIP Team Phoenix--Cluster Competition Team, Faculty advisor
    \item (2021--Present) TSO advisory committee representative
    \item (2021--Present) Seminar series organizer (with \Florian and R.\ Vuduc)
    \item (2022--Present) CSE communication committee
    \item (2022) Organizer, Georgia Scientific Computing Symposium (with E.\ Chow and X.\ Zhang)
    \item (2022) Judge, CS Junior Design Capstone Expo
    \item (2021--22) Graduate student admissions committee
\end{itemize}

\subsubsection{Student examination committees}

\begin{itemize}
    \item (2022) Ph.D. defense; Wangwei Lan (CoS Physics)
    \item (2022) Qualifying exam, Dissertation Proposal; Johnie Sublett (CoC CSE)
    \item (2022) Ph.D. defense; Achyut Panchal (CoE AE)
    \item (2021) Qualifying exam; Bradley Baker (CoC CSE)
    \item (2021) Qualifying exam; Conlain Kelly (CoC CSE)
    \item (2021) Qualifying exam; Sam Swanson (CoC CSE)
\end{itemize}

\subsection{External}

\begin{itemize}
    \item (2021--Present) Mentor, GPU Hackathons (with Oak Ridge National Lab and NVIDIA)
    \item (2022) Supercomputing (SC) Mentor (via Mentor--Protege program)
    \item (2022) Supercomputing (SC) Early Career Program
    \item (2022) Panel Referee, ACCESS Maximize
    \item (2022) Grant Panel Reviewer, National Science Foundation
    \item (2021, 2022) Session chair, American Physical Society, Division of Fluid Dynamics
    \item (2021--22) Research mentor, XSEDE EMPOWER (Expert Mentoring Producing Opportunities for Work, Education, and Research; program received HPCwire 2021 Editors' Choice Award in Workforce Diversity and Inclusion Leadership)
    \item (2021) Poster judge, American Physical Society, Division of Fluid Dynamics
    \item (2021) Mini-symposium organizer and session chair, ``Machine learning for multiphase flows'', IACM Conference on Mechanistic Machine Learning and Digital Twins for Computational Science, Engineering \& Technology (MMLDT-CSET)
    \item (2020) Research mentor, Schmidt Academy for Software Engineering
    \item (2019) Research mentor, WAVE undergraduate research program for under-represented students, Caltech
    \item (2015--16) Judge, Illinois State-wide Math Competition
    \item (2014) Organizer, Science Night, Illinois Middle Schools
\end{itemize}

\section{Publications}

\nocite{*}

\newrefcontext[labelprefix=P]
\printbibliography[type=unpublished,title={Preprints},resetnumbers=true,heading=subbibnumbered]

\newrefcontext[labelprefix=J]
\printbibliography[type=article,title={Journal papers},resetnumbers=true,heading=subbibnumbered]

\newrefcontext[labelprefix=C]
\printbibliography[type=inproceedings,title={Refereed conference papers},resetnumbers=true,heading=subbibnumbered]

\newrefcontext[labelprefix=O]
\printbibliography[title={Other publications},resetnumbers=true,filter=other,heading=subbibnumbered]

\section{Talks}

\newrefcontext[labelprefix=I]
\printbibliography[title={Invited talks},resetnumbers=true,filter=invited,heading=subbibnumbered]

\newrefcontext[labelprefix=T]
\printbibliography[title={Conference presentations},resetnumbers=true,filter=talk,heading=subbibnumbered]

\subsection{Software}

Our software is located at \href{https://github.com/comp-physics}{\texttt{github.com/comp-physics}}, below is an autogenerated listing:
\vspace{-0.5cm}
\begin{center}
\begin{longtable}{r p{3in}}%
    \textbf{Name} (click for Github repo.) & \bfseries Description%
    \csvreader[head to column names]{github-cpg.csv}{}%
    {\\\hline \href{\url}{\texttt{\name}} & \description}%
\end{longtable}
\end{center}
\vspace{-0.75cm}
We maintain \href{https://mflowcode.github.io}{\texttt{MFC}}, an exascale-ready multiphase CFD solver:
\vspace{-0.5cm}
\begin{center}
\begin{longtable}{r p{3in}}%
    \textbf{Name} (click for Github repo.) & \bfseries Description%
    \csvreader[head to column names]{github-mfc.csv}{}%
    {\\\hline \href{\url}{\texttt{\name}} & \description}%
\end{longtable}
\end{center}
\vspace{-0.75cm}
We also work on Inertial Microcaviation Rheometry (IMR) software:
\vspace{-0.5cm}
\begin{center}
\begin{longtable}{r p{3in}}%
    \textbf{Name} (click for Github repo.) & \bfseries Description%
    \csvreader[head to column names]{github-imr.csv}{}%
    {\\\hline \href{\url}{\texttt{\name}} & \description}%
\end{longtable}
\end{center}

% \newrefcontext[labelprefix=S]
% \printbibliography[type=software,title={Software},resetnumbers=true,heading=bibnumbered]

\end{document}
