%!TEX-root = cv.tex
%!TEX TS-program = pdflatex

\documentclass{article}

\usepackage[left=1.5in,right=1.5in,top=1in,bottom=1in]{geometry}
\usepackage{comment}
\usepackage{tabu}
\usepackage{booktabs}
\usepackage{xstring}
\usepackage{etoolbox}
\usepackage[backend=bibtex,defernumbers=true,sorting=ydnt,style=authoryear,doi=true,maxnames=99,url=true]{biblatex}
\usepackage{calc}
\usepackage{comment}
\usepackage{microtype}
\usepackage{hyphenat}
\usepackage{changepage}
\usepackage{multicol}
\usepackage{xspace}
\usepackage{enumitem}
\usepackage{hyperref}
\usepackage[]{xcolor}

\definecolor{lightblue}{rgb}{0.63, 0.74, 0.78}
\definecolor{seagreen}{rgb}{0.18, 0.42, 0.41}
\definecolor{orange}{rgb}{0.85, 0.55, 0.13}
\definecolor{silver}{rgb}{0.69, 0.67, 0.66}
\definecolor{rust}{rgb}{0.72, 0.26, 0.06}

\colorlet{lightsilver}{silver!30!white}
\colorlet{darkorange}{orange!75!black}
\colorlet{darksilver}{silver!65!black}
\colorlet{darklightblue}{lightblue!65!black}
\colorlet{darkrust}{rust!85!black}


\hypersetup{
  colorlinks=true,
  linkcolor=silver,
  citecolor=silver,
  urlcolor=seagreen,
  pdfauthor=author,
}

\newcommand\GIT{\mbox{Georgia Institute of Technology}\xspace}  
\newcommand\GT{\mbox{Georgia Tech}\xspace}  
\newcommand\CIT{\mbox{California Institute of Technology}\xspace}  
\newcommand\MIT{\mbox{Massachusetts Institute of Technology}\xspace}  
\newcommand\UIUC{\mbox{University of Illinois at Urbana--Champaign}\xspace}  
\newcommand\UMD{\mbox{University of Michigan--Dearborn}\xspace}

\newcommand\Florian{\mbox{F. Sch\"{a}fer}\xspace}
\newcommand\Rich{\mbox{R. Vuduc}\xspace}

\newcommand\APBC{\mbox{Assistant Professor by Courtesy (0\%)}\xspace}

% Remove indent spacing
\setlength{\parindent}{0in}

% Filters for bib
\addbibresource{ref.bib}

\defbibfilter{other}{
  type=report or
  type=thesis
}

\defbibfilter{invited}{
  type=incollection and
  keyword=invited
}

\defbibfilter{talk}{
  type=incollection and
  not keyword=invited
}

\defbibfilter{heavy}{
  type=inproceedings and
  keyword=heavyref
}

\defbibfilter{nonheavy}{
  type=inproceedings and
  not keyword=heavyref
}

\defbibfilter{fullpapers}{
    type=article or keyword=heavyref
}

% Bold names
\newboolean{bold}
\newcommand{\makeauthorsbold}[1]{%
  \DeclareNameFormat{author}{%
  \setboolean{bold}{false}%
    \renewcommand{\do}[1]{\expandafter\ifstrequal\expandafter{\namepartfamily}{####1}{\setboolean{bold}{true}}{}}%
    \docsvlist{#1}%
    \ifthenelse{\value{listcount}=1}
    {%
     {\expandafter\ifthenelse{\boolean{bold}}{\mkbibbold{\namepartfamily\addcomma\addspace \namepartgiveni}}{\namepartfamily\addcomma\addspace \namepartgiveni}}%
    }{\ifnumless{\value{listcount}}{\value{liststop}}
      {\expandafter\ifthenelse{\boolean{bold}}{\mkbibbold{\addcomma\addspace \namepartfamily\addcomma\addspace \namepartgiveni}}{\addcomma\addspace \namepartfamily\addcomma\addspace \namepartgiveni}}%
      {\expandafter\ifthenelse{\boolean{bold}}{\mkbibbold{\addcomma\addspace \namepartfamily\addcomma\addspace \namepartgiveni\addcomma\isdot}}{\addcomma\addspace \namepartfamily\addcomma\addspace \namepartgiveni\addcomma\isdot}}%
      }
    \ifthenelse{\value{listcount}<\value{liststop}}
    {\addcomma\space}{}
  }
}
\makeauthorsbold{Le Berre,Bati,Chrit,Lee,Kocherla,Radhakrishnan,Arias,Song,Wilfong,Hawkins,Yu,Panchal,Zeng,Kothari,Sinha,Chu,Manasreh,Gupta,Adam,{*}Wilfong,{*}Radhakrishnan,{*Le Berre}}

% Get reverse numbering
\makeatletter
\patchcmd{\blx@printbibliography}
  {\blx@bibliography\blx@tempa}
  {\setcounter{bibitemtotal}{0}%
   \begingroup
   \def\do##1{\stepcounter{bibitemtotal}}%
   \dolistloop{\blx@tempa}%
   \endgroup
   \blx@bibliography\blx@tempa}{}{}
\makeatother

\newcounter{bibitemtotal}
\newrobustcmd*{\mkbibdesc}[1]{%
  \number\numexpr\value{bibitemtotal}+1-#1\relax}
\DeclareFieldFormat{labelnumber}{\mkbibdesc{#1}}
\DeclareFieldFormat{labelnumberwidth}{\mkbibbrackets{#1}}

\defbibenvironment{bibliography}
  {\list
     {\printtext[labelnumberwidth]{\printfield{labelprefix}\printfield{labelnumber}}}
     {}%
      \renewcommand*{\makelabel}[1]{\hss##1}}
  {\endlist}
  {\item}

% Set spacing in itemize
\setlist{itemsep=3pt,topsep=3pt,parsep=0pt,partopsep=0pt} 

% Set spacing around titles (syntax: spacing left, before, after)
\usepackage{titlesec}
\titlespacing*{\section}
{0pt}{6pt}{6pt}
\titlespacing*{\subsection}
{0pt}{6pt}{6pt}

\usepackage{lastpage}
\usepackage{fancyhdr}
\pagestyle{fancy}
\renewcommand{\headrulewidth}{0pt}

\lhead{}\chead{\color{silver}CV Compiled on: \today}\rhead{}
\cfoot{}
\rfoot{\color{silver}Page \thepage\xspace of \pageref{LastPage}}
\cfoot{\color{silver}Latest at \href{https://comp-physics.group/cv/cv.pdf}{\tt comp-physics.group/cv/cv.pdf}}
\lfoot{\color{silver}Spencer H. Bryngelson}

% \usepackage{titletoc}
% \setcounter{secnumdepth}{1}
% \setcounter{tocdepth}{1}

\usepackage{csvsimple}
\usepackage{longtable}


\usepackage[T1]{fontenc}
\usepackage{ebgaramond}
% \usepackage{garamondlibre}
% \usepackage{garamondx}
% \usepackage{CormorantGaramond}

%% For TOC
\usepackage{tocloft}               % For table of contents formatting
\usepackage{setspace}              % For spacing
% \setcounter{tocdepth}{1}

%% Table of contents formatting
% \renewcommand\cftsecafterpnum{\vskip 0pt} % for spacing after each entry
% \renewcommand\cftsubsecafterpnum{\vskip 0pt} % for spacing after each entry
% \renewcommand\cftsubsubsecafterpnum{\vskip 5pt} % for spacing after each entry
% \setlength{\cftbeforetoctitleskip}{0in}


\begin{document}

\begin{center}
    {\LARGE \bf Spencer H. Bryngelson} 
\end{center}

% \tableofcontents

\section{Basic information}
\begin{itemize}
    \item[] \textbf{Title:} Assistant Professor, School of Computational Science \& Engineering \\
        \phantom{\textbf{Title:}} \APBC, Daniel Guggenheim School of Aerospace Engineering 
        \phantom{\textbf{Title:}} \APBC, George W. Woodruff School of Mechanical Engineering
    \item[] \textbf{Institution:} \GIT
    \item[] \textbf{Office Coordinates:} S1313 CODA, 756 W Peachtree St NW, Atlanta, GA 30308
    \item[] \textbf{Office Phone:} \texttt{(404)894-5821}
    \item[] \textbf{Email:} \href{mailto:shb@gatech.edu}{\texttt{shb@gatech.edu}}
    \item[] \textbf{Website:} \href{https://comp-physics.group}{\texttt{https://comp-physics.group}}
    \item[] \textbf{Citizenship:} USA (natural-born)
\end{itemize}

\section{Education}

\begin{itemize}
    \item \UIUC
    \begin{itemize}
        \item[] (2018) Doctor of Philosophy, Theoretical \& Applied Mechanics \\
            \phantom{(2018)} Advisor: Jonathan Freund
        \item[] (2015) Master of Science, Theoretical \& Applied Mechanics
        \item[] (2015) Graduate Certificate, Computational Science \& Engineering
    \end{itemize}
    \item \UMD
    \begin{itemize}
        \item[] (2013) Bachelor of Science, Mechanical Engineering
        \item[] (2013) Bachelor of Science, Engineering Mathematics
    \end{itemize}
\end{itemize}


\section{Positions held}

\begin{itemize}
    \item (2021--Present) Assistant Professor, School of Computational Science \& Engineering, College of Computing, \GIT
    \item (2023--Present) Assistant Professor by Courtesy (0\%), Daniel Guggenheim School of Aerospace Engineering, College of Engineering, \GIT
    \item (2024--Present) Assistant Professor by Courtesy (0\%), George W.\ Woodruff School of Mechanical Engineering, College of Engineering, \GIT
    \item (2022) Visiting Scholar, Stanford University, Center for Turbulence Research (Summer Program)
    \item (2018--21) Senior Postdoctoral Scholar, \CIT, with Tim Colonius
    \item (2019) Visiting Researcher, \MIT, with Themis Sapsis
    \item (2018) Postdoctoral Researcher, XPACC (PSAAP II center), with Carlos Pantano, Dan Bodony, Jon Freund
    \item (2013--17) Graduate Research Fellow, \UIUC, with Jon Freund
    \item (2015) Alumni Teaching Fellow, \UIUC
    \item (2012--13) Undergraduate Research Assistant, \UMD, with Eric Ratts
\end{itemize}

\section{Teaching}

\subsection{\GIT}

\begin{center}
    \begin{longtable}{ r l l c c }
        \bf Semester&\bf Number     & \bf Course Title                  & \bf Students  \\ \midrule
        Fall 2024   & CX/MATH 4640   & Numerical Analysis I              & \phantom{1}45 \\
        Spring 2024 & CSE 6730       & Modeling \& Simulation            & 163 \\
        Fall 2023   & CX/MATH 4640   & Numerical Analysis I              & \phantom{1}53 \\
        Spring 2023 & CSE 6730       & Modeling \& Simulation            & 146           \\
        Fall 2022   & CX/MATH 4640   & Numerical Analysis I              & \phantom{1}36 \\
        Fall 2021   & CX/MATH 4640   & Numerical Analysis I              & \phantom{1}43 
    \end{longtable}
\end{center}
\vspace{-0.8cm}
\textbf{Note:} I co-teach VIP (2/3/3/4)60(1/2) \textit{Team Phoenix: Cluster Competition Team (HPC)} with Prof.~R.~Vuduc each Fall and Spring since Fall~2022.

\subsection{Other institutions}

\begin{center}
    \begin{tabular}{ r l l c l }
        \bf Semester  &\bf Number & \bf Course Title & \bf Students & \bf Institute \\
        \midrule
        Fall   2015 & ME310  & Fundamentals of Fluid Dynamics & 82 & UIllinois \\
        Fall   2013 & ME3601 & Design and Analysis of Machine Elements & 35 & UMichigan\\
        Spring 2012 & ME364  & Probability, Statistics, and Reliability in Design & 32 & UMichigan \\
        Fall   2012 & ME230  & Statics and Mechanics of Materials & 61 & UMichigan \\
    \end{tabular}
\end{center}

\section{Students}

\subsection{Postdoctoral researchers}

\begin{itemize}
    \item Dr.\ Daehyun Choi (co-advised, primary advisor S.\ Bhamla)
    \item Dr.\ Tianyi Chu
\end{itemize}

\subsection{Ph.D.}

\begin{itemize}
    \item Dimitrios Adam (CSE/AE)
    \item Jesus Arias (CSE/AE)
    \item Max Hawkins (CSE), co-advised with \Rich
    \item Anand Radhakrishnan (CSE)
    \item Zhixin Song (Physics)
    \item Benjamin Wilfong (CSE)
    \item Haocheng Yu (CSE/AE), co-advised with K.\ Ahuja
\end{itemize}

\subsection{Undergraduate}

\begin{itemize}
    \item Ansh Gupta (CS)
    \item Elizabeth Hong (CS), co-advised with \Rich
    \item Henry Le Berre (CS)
    \item Melody Lee (CS)
    \item Brian Ok (CS)
    \item Lian Xiang (Physics)
\end{itemize}

\subsection{Alumni}

\subsubsection{Graduate students}

\begin{itemize}
    \item Fatima Ezahra Chrit, Ph.D.\ ME, co-advised with Alex Alexeev, 2021--23
    \item Anshuman Sinha, M.S.\ CSE, 2022--23
\end{itemize}

\subsubsection{Undergraduate students}

\begin{itemize}
    \item Ajay Bati, CS, 2021--23
    \item Arjun Bhamra, CS, 2022--23
    \item Rasmit Devkota, Physics, 2023
    \item Yash Kothari, CS, 2022--23
    \item Suzan Manasreh, CS, 2024
    \item Sriharsha Kocherla, CS, 2022-24
    \item Subrahmanyam Mullangi, CS, 2023--24
    \item Qi Zeng, CS and Math, co-advised with \Florian, 2021--23
\end{itemize}

\subsection{Student and scientist accolades}

\begin{itemize}
    \item (2024) Benjamin Wilfong, GT CRNCH Fellowship
    \item (2024) Elizabeth Hong, GT PURA Salary Award
    \item (2024) Suzan Manasreh, GT PURA Salary Award
    \item (2024) Subrahmanyam Mullangi, GT PURA Salary Award
    \item (2023) Dr.\ Bryan Gard (GTRI Research Scientist), IRAD of the Year award
    \item (2023) Qi Zeng, Outstanding Undergraduate Researcher Award, College of Computing (co-advised with \Florian)
    \item (2023) Ansh Gupta, GT PURA Salary Award
    \item (2022) Fatima Chrit, Georgia Tech Quantum Alliance Fellowship
    \item (2022) Zhixin (Jack) Song, GT CRNCH Fellowship
    \item (2022) Benjamin Wilfong, GT President's Fellowship
\end{itemize}

\section{Awards}

\begin{itemize}
    \item (2022) Ralph E. Powe Junior Faculty Enhancement Award, Oak Ridge National Lab
    \item (2022--23) \GT Faculty Writing Scholar
    \item (2022--23) Class of 1969 Teaching Fellow, \GIT
    \item (2018) Stanley Weiss Outstanding Dissertation Award, \UIUC
    \item (2016) Hassan Aref Award (research in fluid mechanics), \UIUC
    \item (2015) Alumni Teaching Fellowship, \UIUC
    \item (2010--13) Dean's List, \UMD
    \item (2011) Pi Tau Sigma (honor society, member), \UMD
\end{itemize}

\section{Research support}

% \subsection{Pending grants}

\begin{itemize}
    \item (2024--29) PI: DOE Early Career Research Program ``\textit{Radial basis function numerics enable massively parallel, high-order accurate, mesh-free, and shock-stable PDE solvers}'' ($\$875$K)
    \item (2024--29) PI: NSF CAREER ``\textit{From Remodeling to Stroke, Vasculature to Cell: Unraveling the mechanics of sickle cell disease}'' ($\$544$K)
    \item (2024--27) PI: DOD AFOSR YIP ``\textit{Actuator design and control in discontinuous flow fields}'' ($\$450$K)
    \item (2024--27) PI: DOD ONR YIP ``\textit{Systematic model improvement for reliably predictive simulations of flow-coupled solid-fuel jet engines}'' ($\$750$K)
\end{itemize}


\subsection{Funded grants}

\subsubsection{Current}

\begin{itemize}
    \item (2024--29) Senior personnel: DARPA HR-0011-472506 ``\textit{Squid-inspired nozzles for enhanced  efficiency and thrust in rotary propulsors}'' ($\$1$M), PI:~S.~Bhamla (GT), SHB~Share: $\$150$K Y1--2
    \item (2023--28) co-PI: DOD ONR MURI N00014-23-1-2501, ``\textit{Combustion of solid fuels in high enthalpy flow}'' ($\$3.8$M) PI:~G.~Young (Virginia Polytechnic Institute and State University), 7 other co-PIs. SHB Share: $\$270$K Y1--3
    \item (2024--27) co-PI: DOD ONR N00014-24-1-2094 ``\textit{Multi-scale simulations of combustion in a solid propellant ramjet with embedded reactive metal particles}'' ($\$375$K), PI:~S.~Menon (GT), SHB~Share: $\$188$K 
    \item (2023--27) PI: DOD ARO W911NF-23-10324, ``\textit{Investigation and inference of soft material deformation mechanisms unlocked at large speeds, finite deformations, and many cycles},'' collaborative with University of Michigan, Jon Estrada. (Total: $\$835$K; SHB Share: $\$314$K)
    \item (2022--26) PI: DOD ONR N00014-22-12519, ``\textit{Stochastic framework for cavitating flows: mesoscale modeling and acceleration}'' ($\$560$K)
    \item (2024--25) PI: DOE DE-NA0003525 (Sandia National Laboratories subcontract), ``\textit{Vibrated bubbly flow simulation}'' ($\$113$K)
\end{itemize}

\subsubsection{Completed}

\begin{itemize}
    \item (2024) PI: DOE DE-AC52-07NA27344 (Lawrence Livermore National Laboratories subcontract), ``\textit{Accelerated, Compressed, and Regularized Computation of Kinetic-based PDEs}'' ($\$80$K)
    \item (2023--24) PI: DOE DE-NA0003525 (Sandia National Laboratories subcontract), ``\textit{Vibrated bubbly flow simulation}'' ($\$100$K)
    \item (2023--24) co-PI: DARPA HR-0011-2330006, ``\textit{Quantum eigensolvers in fluid-dynamic computations and applications}'' ($\$300$K), PI B.~Gard (Georgia Tech Research Institute), SHB~Share:~$\$100$K
    \item (2022--23) PI: DOE DE-NA0003525 (Sandia National Laboratories subcontract), ``\textit{Vibrated bubbly flow simulation}'' ($\$65$K)
    \item (2022--23) PI: DOE ORAU Powe, ``\textit{A methodologically coherent multi-scale model for multiphase flow}'' ($\$10$K)
    \item (2022--23) co-PI: GTRI IRAD, ``\textit{Quantum optimization for lattice Boltzmann simulation (QOLBS)}'' ($\$40$K), PI:~B.~Gard (Georgia Tech Research Institute)
    \item (2022) PI: GT Seed Grant, Forming Teams ``\textit{Quantum computing for next-generation engineering simulation}'' ($\$50$K)
    \item (2022) PI: GTQA DE-00013211, ``\textit{Quantum algorithms for lattice Boltzmann fluid flow simulation}'' ($\$14.5$K)
\end{itemize}

\subsection{Funded resource and hardware awards}

\begin{itemize}
    \item (2024--Present) PI: ACCESS-CI Maximize TG-PHY240200, ``\textit{Direct simulation of compressible multiphase flow}'' (225K~GPU Hours, 55K~CPU Hours, $\$119$K value)
    \item (2024--Present) PI: ACCESS-CI Accelerate TG-PHY210084, ``\textit{High-fidelity simulation of high-flowing dispersions}'' (3M~ACCESS Credits, $\$24$K value)
    \item (2024--Present) PI: J\"ulich Supercomputing Center, JUPITER Exascale Early Access Program, ``\textit{ExaMFlow: Exascale simulation enables multiphase flow simulation at the finest scales}''
    \item (2021--Present) PI: Oak Ridge National Lab CFD154, Director's Discretionary, ``\textit{Accelerated sub-grid multi-component flow physics}'' (100K~node hours+)
    \item (2021--2023) PI: ACCESS-CI Discovery TG-PHY210084, ``\textit{High-fidelity simulation of high-speed flowing dispersions via a stochastic sub-grid model}'' (10K~GPU Hours, 20K~CPU Hours, $\$7.5$K value)
    \item (2024) co-PI: Georgia Tech Tech.\ Fee, ``\textit{Next Generation NVIDIA HPC Cluster}'' (4x NVIDIA GraceHopper Superchip nodes, $\$250$K)
    \item (2022) PI: NVIDIA Academic Hardware Grant Program (4x BlueField-2 E-Series DPU, $\$12$K value)
    \item (2022) PI: Georgia Tech Tech.\ Fee ``\textit{ARM HPC Dev Kits for next-generation supercomputing}'' (10x NVIDIA ARM HPC Dev.\ Kits, $\$240$K)
    \item (2022) PI: AMD MI200-series GPU Server ($\$77$K value)
    \item (2022) PI: NVIDIA Academic Hardware Grant Program (2x A100 80GB PCIe GPUs, $\$30$K value)
    \item (2019--20) co-PI: XSEDE TG-CTS120005, ``\textit{Advanced immersed boundary and interface-capturing methods for simulations of complex flows}'' (9M CPU hours, $\$71$K value)
\end{itemize}

\subsection{Other awarded funds}

\begin{itemize}
    \item (2023) PI: SIAM CSE Travel Award ($\$1$K)
    \item (2023) PI: APS FECS Travel Grant ($\$350$)
    \item (2022) PI: Stanford CTR Summer Program ``\textit{Fast macroscopic forcing for operator recovery via locality and causality with application to compressible and multiphase flow}'' ($\$8$K, with \Florian, SHB share: $\$4$K)
\end{itemize}

% \subsection{Rejected proposals and awards}

\begin{itemize}
    \item (2024--27) PI: NSF DARE ``\textit{Optimal computational model-based design of affordable wearable technology to monitor biomarkers in kids with enthesitis related arthritis}'' ($\$450$K)
    \item (2024--25) PI: Google Research Scholar Program ``\textit{Fluid flow solver congruent with current quantum devices}'' ($\$60$K)
    \item (2023--26) PI: DOD AFOSR YIP ``\textit{Actuator design and control in discontinuous flow fields}'' ($\$450$K)
    \item (2023--28) PI: NSF CAREER ``\textit{From Remodeling to Stroke, Vasculature to Cell: Unraveling the mechanics of sickle cell disease}'' ($\$526$K)
    \item (2023--28) PI: DOE Early Career Research Program ``\textit{Massively parallel mesh-free hyperbolic PDE solvers via adaptive radial-basis-function-based numerics}'' ($\$750$K)
    \item (2023--26) PI: DOD ONR YIP ``\textit{Systematic model improvement for reliably predictive simulations of flow-coupled solid-fuel jet engines}'' ($\$750$K)
    \item (2023--26) PI: NSF DARE ``\textit{Optimal computational model-based design of affordable wearable technology to monitor biomarkers in kids with enthesitis related arthritis}'' ($\$450$K)
    \item (2023--25) PI: DOE INCITE, ``\textit{The first full-resolution liquid--gas disperse flow simulations}'' (500K Summit Node Hours, 2M Frontier Node Hours, $\$$1M value)
    \item (2023--24) co-PI: NOAA SBIR ``\textit{Using bubbles to reduce underwater noise from shipping and ferries}'' ($\$175$K, SHB share: $\$24.5$K, PI K.\ Seger, Applied Ocean Sciences)
    \item (2023--24) PI: GT Seed Grant, Moving Teams Forward ``\textit{Quantum computing for next-generation engineering simulation}'' ($\$100$K)
    \item (2023--24) PI: Google Research Scholar Program ``\textit{Solving partial differential equations on noisy quantum processors}'' ($\$60$K)
    \item (2023--24) PI: ASME Hawthornwaite Research Initiation Grant ($\$20$K)
    \item (2023--24) PI: Google Collabs, ``\textit{Enabling subgrid-scale modeling of multiphase flows using differentiable CFD and machine learning models}'' ($\$90$K)
    \item (2022--25) co-PI: NSF OAC CORE 21-616, ``\textit{Enabling rapid, targeted optimization of PDE solvers via hardware--software coupled autotuning for novel architectures}'' (SHB share: $\$300$K, Total: $\$600$K)
    \item (2022--24) PI: DOE ASCR DE-FOA-0002717, ``\textit{Split, Prune, Unify: A quantum algorithm development strategy}'' (SHB share: $\$200$K, Total: $\$400$K)
    \item (2022--24) co-PI: DOE ASCR DE-FOA-0002717, ``\textit{Sharing and combining SciML models, robustly and securely}'' (SHB share: $\$200$K, Total: $\$400$K)
    \item (2022--23) co-PI: NOAA SBIR OAR-TPO 2007117 ``\textit{Using bubbles to reduce underwater noise from ships and ferries}'' (SHB share: $\$39$K, Total: $\$125$K)
\end{itemize}


\section{Professional activity}

\subsection{Appointments and memberships}

\begin{itemize}
    \item (2024--Present) University Consortium for Applied Hypersonics (UCAH)
    \item (2022--Present) Association for Computing Machinery (ACM), Member
    \item (2021--Present) NATO Science \& Technology Organization, Technical Team Member
    \item (2021--Present) American Institute of Aeronautics \& Astronautics (AIAA), Member
    \item (2015--Present) Society of Industrial and Applied Mathematics (SIAM), Member
    \item (2014--Present) American Physical Society (APS), Member
\end{itemize}

\subsection{Referee}

\subsubsection{Journals and Conferences}

% AIAA Journal,
% Computers and Fluids,
% Computers in Biology and Medicine,
% Computer Methods in Applied Mechanics and Engineering,
% Fluids,
% IEEE International Parallel \& Distributed Processing Symposium,
% International Journal of Multiphase Flow,
% International Journal of Offshore and Polar Engineering,
% Journal of Computational Physics,
% Journal of Computational Science,
% Journal of Fluid Mechanics,
% Nature Communications Physics,
% PEARC (Practice and Experience in Advanced Research Computing),
% Physical Review A,
% Physical Review E,
% Physical Review Fluids,
% PLOS Computational Biology,
% SC (International Conference for High Performance Computing, Networking, Storage, and Analysis),
% SIAM Multiscale Modeling \& Simulation,
% SIAM Scientific Computing,
% Soft Matter,
% SoftwareX,
% Symposium of Naval Hydrodynamics,
% Theoretical and Computational Fluid Dynamics

\vspace{-0.5cm}
\begin{multicols}{2}
\begin{itemize}
    \item AIAA Journal
    \item Computers and Fluids
    \item Computers in Biology and Medicine
    \item Computer Methods in Applied Mechanics and Engineering
    \item Fluids
    \item IEEE International Parallel \& Distributed Processing Symposium
    \item International Journal of Multiphase Flow
    \item International Journal of Offshore and Polar Engineering
    \item Journal of Computational Physics
    \item Journal of Computational Science
    \item Journal of Fluid Mechanics
    \item Nature Communications Physics
    \item PEARC (Practice and Experience in Advanced Research Computing)
    \item Physical Review A
    \item Physical Review E
    \item Physical Review Fluids
    \item PLOS Computational Biology
    \item SC (International Conference for High Performance Computing, Networking, Storage, and Analysis)
    \item SIAM Multiscale Modeling \& Simulation
    \item SIAM Scientific Computing
    \item Soft Matter
    \item SoftwareX
    \item Symposium of Naval Hydrodynamics
    \item Theoretical and Computational Fluid Dynamics
\end{itemize}
\end{multicols}

\subsubsection{Research proposals}

Israel Science Foundation, US Department of Defense (Army Research Office), US National Science Foundation (ENG)

\section{Service and outreach}

\subsection{Georgia Tech}

\subsubsection{Institute-level}

\begin{itemize}
    \item (2021--Present) Georgia Tech \textit{HPC Hackathon}, initiator and organizer, recruited sponsors Oak Ridge National Lab and NVIDIA 
    \item (2022--Present) Georgia Tech \textit{Scientific Software Engineering Center}, Advisory Board
    \item (2022--Present) PURA Award Reviewer
    \item (2024) Schmidt Science Polymaths Award Reviewer
    \item (2022,2023) ORAU Powe Award Reviewer
    \item (2022) Faculty Search Panel, Professional Development Workshops, Georgia Tech Center for Teaching and Learning
\end{itemize}

\subsubsection{College-level}

\begin{itemize}
    \item (2024--Present) Modeling \& Simulation, School of CSE, Area lead
    \item (2022--Present) VIP Team Phoenix--Cluster Competition Team, Faculty advisor
    \item (2021--Present) TSO advisory committee representative
    \item (2022--24) CSE communication committee
    \item (2021--24) Seminar series organizer (with \Florian)
    \item (2023) Computational Mathematics Activity Group (organized by N.\ Chandramoorthy)
    \item (2023) CRNCH Summit Panel organizer and moderator (with \Rich)
    \item (2022) Organizer, Georgia Scientific Computing Symposium (with E.\ Chow and X.\ Zhang)
    \item (2022) Judge, CS Junior Design Capstone Expo
    \item (2021--22) Graduate student admissions committee
\end{itemize}

\subsubsection{Student examination committees}

\textit{Ph.D.\ Thesis defense}
\vspace{-0.25cm}
\begin{multicols}{2}
\begin{itemize}
    \item (2024)  Hohyun Lee (CoE ME)
    \item (2023)  Fatima Ezahra Chrit (CoE ME)
    \item (2022)  Achyut Panchal (CoE AE)
    \item (2022)  Wangwei Lan (CoS Physics)
\end{itemize}
\end{multicols}

\textit{Ph.D.\ Thesis proposal}
\vspace{-0.25cm}
\begin{multicols}{2}
\begin{itemize}
    \item (2024)  Micaiah Smith-Pierce (CoE AE)
    \item (2024)  Sara Karamati (CoC CSE)
    \item (2023)  Liana Hatoum (CoE BME)
    \item (2022)  Johnie Sublett (CoC CSE)
\end{itemize}
\end{multicols}

\textit{Ph.D. Qualifying examination}
\vspace{-0.25cm}
\begin{multicols}{2}
\begin{itemize}
    \item (2024)  Srikanth Avasarala (CoC CSE)
    \item (2024)  Benjamin Wilfong (CoC CSE)
    \item (2024)  Jesus Arias (CoC CSE)
    \item (2024)  Lynn Jin (CoS Physics)
    \item (2024)  Sijian Tan (CoE AE)
    \item (2023)  Ayush Jain (CoC CSE)
    \item (2023)  Hohyun Lee (CoE ME)
    \item (2023)  Grayson Harrington (CoC CSE)
    \item (2022)  Anand Radhakrishnan (CoC CSE)
    \item (2022)  Johnie Sublett (CoC CSE)
    \item (2021)  Bradley Baker (CoC CSE)
    \item (2021)  Conlain Kelly (CoC CSE)
    \item (2021)  Sam Swanson (CoC CSE)
\end{itemize}
\end{multicols}

\textit{Other}
\vspace{-0.1cm}
\begin{itemize}
    \item (2023) M.S.\  Thesis defense;  Felix Luo (CoE AE)
    \item (2023) M.S.\  Thesis proposal; Felix Luo (CoE AE)
\end{itemize}

\subsection{External}

\begin{itemize}
    \item (2024) Session chair, International Conference on Theoretical and Applied Mechanics
    \item (2024) Session chair, International Conference on Numerical Methods in Multiphase Flows
    \item (2024) Sorting committee, American Physical Society, Division of Fluid Dynamics
    \item (2022--Present, bi-annual) Panel Referee, ACCESS-CI Maximize
    \item (2021--Present, annual) Mentor, GPU Hackathons (with Oak Ridge National Lab, NVIDIA, NASA)
    \item (2021,22,24) Session chair, American Physical Society, Division of Fluid Dynamics
    \item (2021,23) Poster judge, American Physical Society, Division of Fluid Dynamics
    \item (2023) Mini-symposium organizer and session chair, ``Statistical Approaches to Closure Modeling in Computational Mechanics,'' IACM Conference on Mechanistic Machine Learning and Digital Engineering for Computational Science, Engineering \& Technology (MMLDT-CSET)
    \item (2023) Session chair, 11th International Conference on Multiphase Flow
    \item (2022) Supercomputing (SC) Mentor (via Mentor--Protege program)
    \item (2022) Supercomputing (SC) Early Career Program
    \item (2021--22) Research mentor, XSEDE EMPOWER (Expert Mentoring Producing Opportunities for Work, Education, and Research; program received HPCwire 2021 Editors' Choice Award in Workforce Diversity and Inclusion Leadership)
    \item (2021) Mini-symposium organizer and session chair, ``Machine learning for multiphase flows,'' IACM Conference on Mechanistic Machine Learning and Digital Twins for Computational Science, Engineering \& Technology (MMLDT-CSET)
    \item (2020) Research mentor, Schmidt Academy for Software Engineering
    \item (2019) Research mentor, WAVE undergraduate research program for under-represented students, Caltech
    \item (2015, 2016) Judge, Illinois State-wide Math Competition
    \item (2014) Organizer, Science Night, Illinois Middle Schools
\end{itemize}

\section{Media}

\begin{itemize}
    \item (2024) Researchers Blazing New Trails with Superchip Named After Computing Pioneer [\href{https://www.cc.gatech.edu/news/researchers-blazing-new-trails-superchip-named-after-computing-pioneer}{\tt LINK}] 

    \item (2023) GTRI, Georgia Tech Use Quantum Computing to Optimize CFD Applications
 [\href{https://gtri.gatech.edu/newsroom/gtri-georgia-tech-use-quantum-computing-optimize-cfd-applications}{\tt LINK}]

    \item (2023) Group Optimizes Fluid Dynamics Simulator on World's Fastest Supercomputer [\href{https://www.cc.gatech.edu/news/group-optimizes-fluid-dynamics-simulator-worlds-fastest-supercomputer}{\tt LINK}]

    \item (2023) Researchers Optimize HPC Software at Interdisciplinary Hackathon
[\href{https://www.cc.gatech.edu/news/researchers-optimize-hpc-software-interdisciplinary-hackathon}{\tt LINK}]

    \item (2022) New Hardware Brings Students Closer to Exascale Computing [\href{https://www.cc.gatech.edu/news/new-hardware-brings-students-closer-exascale-computing}{\tt LINK}]

    \item (2022) Faculty Receives New GPUs for Fluid Dynamics and Machine Learning Research [\href{https://www.cc.gatech.edu/news/faculty-receives-new-gpus-fluid-dynamics-and-machine-learning-research}{\tt LINK}]
\end{itemize}

\section{Publications}

\begin{center}
    \textit{Bolding indicates advised or co-advised students and postdocs.}
\end{center}

\nocite{*}

\newrefcontext[labelprefix=PP]
\printbibliography[type=unpublished,title={Preprints},resetnumbers=true,heading=subbibnumbered]

\newrefcontext[labelprefix=P]
\printbibliography[title={Archival, heavily refereed papers},resetnumbers=true,filter=fullpapers,heading=subbibnumbered]

\newrefcontext[labelprefix=C]
\printbibliography[title={Conference papers},resetnumbers=true,filter=nonheavy,heading=subbibnumbered]

\newrefcontext[labelprefix=O]
\printbibliography[title={Other published content},resetnumbers=true,filter=other,heading=subbibnumbered]

\section{Talks}

\newrefcontext[labelprefix=I]
\printbibliography[title={Invited talks},resetnumbers=true,filter=invited,heading=subbibnumbered]

\newrefcontext[labelprefix=T]
\printbibliography[title={Conference presentations},resetnumbers=true,filter=talk,heading=subbibnumbered]

\section{Software}
We develop and maintain \href{https://mflowcode.github.io}{\texttt{MFC}}, an exascale multiphase and multiphysics fluid flow solver:
\vspace{-0.5cm}
\begin{center}
    {\small
    \def\arraystretch{1.1}
    \begin{longtable}{r p{3in}}%
        \normalsize \textbf{Name} (click for Github repo.) & \normalsize\bfseries Description%
        \normalsize \csvreader[head to column names]{github-mfc.csv}{}%
        {\\ \href{\url}{\texttt{\name}} & \description}%
    \end{longtable}
    }
\end{center}
\vspace{-0.5cm}
More generally, our open source software is located at \href{https://github.com/comp-physics}{\texttt{github.com/comp-physics}}, below is an autogenerated listing:
\vspace{-0.5cm}
\begin{center}
    {\small
    \def\arraystretch{1.1}
    \begin{longtable}{r p{3in}}%
        \normalsize \textbf{Name} (click for Github repo.) & \normalsize\bfseries Description% 
        \csvreader[head to column names]{github-cpg.csv}{}%
        {\\ \href{\url}{\texttt{\name}} & \description}%
    \end{longtable}
}
\end{center}
\vspace{-0.5cm}
We also work on Inertial Microcaviation Rheometry (IMR) software:
\vspace{-0.5cm}
\begin{center}
    {\small
    \def\arraystretch{1.1}
    \begin{longtable}{r p{3in}}%
        \normalsize \textbf{Name} (click for Github repo.) & \normalsize\bfseries Description%
        \normalsize \csvreader[head to column names]{github-imr.csv}{}%
        {\\ \href{\url}{\texttt{\name}} & \description}%
    \end{longtable}
    }
\end{center}

% \newrefcontext[labelprefix=S]
% \printbibliography[type=software,title={Software},resetnumbers=true,heading=bibnumbered]

\end{document}
